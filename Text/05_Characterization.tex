
\section{Characterization of a digital Signal}
% TODO continuous to digital

\subsection{Envelope of a digital signal}

Despite being an elusive concept, the temporal amplitude envelope of a signal is essential for its complete characterization, being the primary information-carrying medium in spoken voice and telecommunications, for example. Intuitively, the temporal envelope can be understood as a smooth function that multiplies the signal, being responsible for its outer shape. 
Envelope detection techniques have applications in areas like health, sound classification and synthesis, seismology and speech recognition. Nevertheless, a general method to digital envelope detection of signals with rich spectral content doesn't exist, as most methods involve manual intervention, in the form of filter design, smoothing, as well as other specific design choices, based on a priori knowledge about the nature of the specific waves under investigation.
To address this problem, we propose a framework that uses intrinsic characteristics of a signal to estimate its envelope, completely eliminating the necessity of parameter tuning.
The approach here described draws inspiration from geometric concepts to isolate the frontiers and thus estimate the temporal envelope of an arbitrary signal; to that end, alpha-shapes, concave hulls, and discrete curvature are explored. We also define entities, such as a pulse and frontiers, in the context of an arbitrary digital signal, as a means to reduce dimensionality and the complexity of the proposed algorithm. Specifically, a new measure of discrete curvature is used to obtain the average radius of curvature of a discrete wave, serving as a threshold to identify the wave's frontier points. 
We find the algorithm accurate in the identification of the frontiers of a wide range of digital sound waves with very diverse characteristics, while localizing each pseudo-cycle of the wave in the time domain. The algorithm also compares favourably with classic envelope detection techniques based on filtering and the Hilbert Transform.
Besides the most direct applications of this work to audio classification and synthesis, we foresee impact in compression techniques and machine learning approaches to audio. The discrete curvature definition presented could also be extended to three-dimensional settings, to improve shape detection algorithms based on alpha-shapes.

\subsubsection{Envelope Detection}
Envelope detection is ubiquitous in both analogue and digital signal processing \parencite{2011CaetanoImproved}. Nevertheless, the literature in this area is very fragmented \parencite{2017LyonsDigital}. Besides, most envelope detection techniques are designed to account for very specific settings, like pure sinusoids with moderate noise content, a limitation that excludes most physical signals, as is the case of recorded sound, for example; that limitation arises in part due to the lack of a strict mathematical definition of a temporal envelope \parencite{2013Mengempirical}.

In many natural signals, however, the temporal amplitude envelope of a signal plays a prominent role in the characteristics exhibited: according to \textcite{2017QiRelative}, for example, the envelope is at least as important as the fine structure of a sound wave in the context of the intelligibility of mandarin tones. This is also the case for English \parencite{1995ShannonSpeech}, where even envelopes modulating mostly noise were still capable of conveying meaning. 

The envelope helps to impart emotion and identity to the human voice \parencite{2018ZhuContributions}, and envelope preserving characteristics in concert halls are associated with their pleasantness \parencite{2011LokkiEngaging}.

When dealing with broadband signals, approaches tailored to specific applications are prevalent, such as the one presented by \textcite{2014YangFast} for the distributed monitoring of fibre optic or the one formulated by \textcite{2018AssefModeling} in the context of medical ultrasound imaging.

Addressing the different units in the horizontal and vertical axes, one can transform the DSP problem of envelope detection into the geometric problem of defining the shape of a set of points in $ \mathbb{R}^2 $. For that purpose, \textcite{1983Edelsbrunnershape} introduced the concept of alpha-shapes, a mathematically well-defined extension to the convex hull of a finite set of points, closely related to the Delaunay triangulation and Voronoi diagrams of those points.

This approach is used in areas such as the detection of features in images \parencite{2016VarytimidisAlpha}, reconstruction of surfaces from a cloud of points \parencite{2015WuAutomated} and Spectroscopy \parencite{2019XuModeling}, with the last work, which involves the estimation and removal of the Blaze function (a kind of envelope) of an echelle spectrograph, being particularly illustrative of the potential of cross-fertilization.

Steps in the direction of translating geometric algorithms to the context of envelope detection were also made by \textcite{2015YangSkeleton} via an algorithm based on the construction of a skeleton underlying the wave, and also via the direct translation of computer vision methods to the task \parencite{2015YangRepresenting}.

Following this path, we present a general approach to envelope detection, exploiting the intrinsic characteristics of a generic, spectrally complex wave, in order to avoid the need for manual intervention or parameter tuning.

While the robustness of the proposed approach allows it to be used as a plug-in replacement for many methods encountered in the literature, we feel that it would be particularly useful for sound synthesis.

The envelope is shown to add complexity to the spectral representation of a wave \parencite{2019TarjanoNeuro}, and an accurate description of the envelope, while describing the evolution of the instantaneous amplitude of a signal in time, would also greatly simplify further spectral analysis. Moreover, the algorithm developed in this work naturally divides a signal into its pseudo-cycles, pinpointing them in the time domain, providing the building blocks for the reconstruction of the fine structure of the wave.

% The rest of this paper is divided as follows: After explaining and defining the various entities to be used subsequently, of which the concept of frontiers is arguably the most important, in the methodology section, we dedicate a section to present a new discrete curvature estimation approach, and how it was used to identify the frontiers of a discrete wave.

We then proceed to illustrate the results of the application of the algorithm in a set of six diverse sound waves, ranging from voice to unpitched instruments comparing, afterwards, its performance with that of classic digital envelope estimation algorithms, both from a visual and numerical standpoint.

We then discuss the implications of the work, emphasizing the focus on the construction of new interpretations of the relationship between time domain and frequency domain sound descriptions, and our vision of the impact of this methodology in the field of sound synthesis, providing directions for future developments.


\subsection{Frequency content}


\subsection{Representations}

% one hot