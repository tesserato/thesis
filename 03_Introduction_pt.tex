\documentclass[12pt]{article}
\usepackage[a4paper, top=3cm, left=3cm, bottom=2cm, right=2cm]{geometry}
\usepackage[utf8]{inputenc}
% \usepackage{fontspec}
% \setmainfont{Arial}
\title{Sound Synthesis}
\author{Carlos Tarjano}
\usepackage[portuguese]{babel}

\begin{document}

\section{Introduction}
% Tell the reader the problem you are tackling in this project.
 
% State clearly how you aim to deal with this problem. 
% Limit the scope of your study. 
% Sketch out how the thesis is structured to achieve your aim. 

% objective
Ambitious as it may seem, it is the goal of this monograph to motivate a change of approach in domains that, despite their modest intersections, share more attributes than is apparent at a superficial glance: Both machine learning and musical acoustics are areas that, despite their current relevance, are fresh out of their respective infancies, and a scrutiny of both promptly reveals the difficulties that arise from this state of affairs. A lack of their own terminology, with frequent borrows from other, more established areas, can be readily cited, as the pronounced prominence of few individual contributions.

Those shortcomings have, however, a bright side to them, in that they expose those areas to reinterpretation, rendering them somewhat open to the influence of outside ideas.

Musical acoustics borrows heavily from the digital signal processing terminology, the last having its roots in the analogic world. Many algorithms are conceived in terms of filters, circuits and other similar legacy constructs, a fact that burdens the conceptual framework of the area. Besides the introduction of a noveau envelope extraction method, we find that his formulation in terms of (differential) geometric concepts is, perhaps, as great a contribution as the algorithm per se.

Perhaps less daring, and owing more to the advanced age of the author than to his research prowess, is the presentation of this monograph in a format more akin to a book; In addition to turning this somewhat dense tome into a more palatable experience. Besides, he whom maketh a monograph wishes not to repeat the experience, save if a new challenge presents itself.
scope
structure
socioeconomical relevance

\end{document}
